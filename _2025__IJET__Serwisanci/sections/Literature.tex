%Theory
\section{Literature}\label{sec:literature} %Done!

There are many studies about VRP in maintenance technician routing applications. However, to the best of our knowledge, there are only a few works that take into account time windows, differentiation of employee rates and the transport of consumables.

Firstly, the works most similar to ours will be presented, containing the VRP extensions we present. One of the most similar problem considering all extensions introduced in our work is given by Zhao et al. \cite{Zhao} in application of aircraft maintenance. The problem is solved with CPLEX, EGA-LNS and Genetic Algorithm on real datasets from Beijing Capital International Airport. Results show that it is not possible to compute the problem using solver due to memory overflow and that results obtained with EGA-LNS are comparable to GA, but with longer computation time. Another example of VRP with Time-Windows, Capacity and skills written by \cite{Huang} et al. consider non-emergency patient transport to hospitals. Differently to us, the authors prohibit the service of certain patients by employees with insufficient skills and consider multi-trip of a vehicle and lunch break. Other paper presented by Yan et al. \cite{Yan} takes into account VRP extension with dynamic service times and different rates depending on skill. To solve the issue, authors use Dynamic Neighbourhood Search algorithm and compare to the results obtained with CPLEX. The experiments carried out took into account a different number of technicians, two objective functions: cost- and time-oriented and the examination of the maximum instance size. 

Next, the focus is on works describing the variable skills and rates of employees, because this is the most valuable feature of this study. In Kiani et al. \cite{Kiani} the authors are modeling VRP with soft time windows and solve them with Genetic Algorithm and Particle Swarm Optimization. What's different to our approach is that the service team is not associated with the vehicle and skills of the technician group does not only affect the cost (wages), but define if a group can service given location. Their conclusion is that in general GA outperforms PSO algorithm, comparing to results obtained with CPLEX. Another example of skill dependent routing problem is given by Schwarze et al. \cite{Schwarze} vehicle skills affect the scope of localizations that can be served by this vehicle. However, in their model, the costs are skill dependent but they affect the cost of the route traveled and not the reduction of service time. For the experiments, modified Solomon instances were used, as in our case, and solved with IBM ILOG CPLEX environment. Carried experiments show, that in  both cases: minimizing total cost and total route length, applying CPLEX solver may improve vehicle load balancing. Similar assumptions are made in Zuo et al. \cite{Zuo} paper, in which however, the problem Solomon's datasets are solved with proprietary heuristics combining Tabu Search and Simulated Annealing algorithms. The conclusions are that the use of multi-skilled employees can reduce the overall cost of enterprise and that given algorithm is suitable for solving such problems.

Finally, the papers analyzing VRP problems with time windows in maintenance applications were also reviewed, as some of the concepts described therein may serve as inspiration for this work. A work by Toru and Yilmaz \cite{Toru} solve Multi-Depot VRP with Time Windows on the example of Turkish compressor manufacturer. The problem is optimized with Gurobi solver and Clustering Algorithm, which returns worse results, however, with much shorter calculation time. Similarly, Vincezna et al. \cite{vincenza} consider VRPTW in maintenance of energy power plants application. In their model, crew skills are not included, however the vehicle capacity is added as the problem assumes pick-up and delivery of goods. Optimization is performed with Large Neighbourhood Search (LNS) and Simulated Annealing algorithm on real routes. Daily duties of maintenance teams are on the other hand optimized in Cassettari et al. \cite{cassettari} by multiple criteria using proprietary heuristics. Their model assumes working time limitation, tasks precedence and limited number of vehicles.

All things considered, there are many works analyzing the combination of various employee skills factor with time windows as an extension of VRP. none of them however, to the best of our knowledge, includes the additional delivery of consumables and and shorten service time depending on the skill. The most similar works \cite{Zhao}, \cite{Huang}, \cite{Yan}, constitute a good starting point and knowledge base for further considerations. This work seems to fill the gap in research by taking into account additional properties of the problem encountered in practice, such as the mentioned transport of consumables, shortened service time for more experienced employees, shift work and time windows.

%*****************************************

