% FIRST section - INTRODUCTION 

\section{Introduction}\label{sec:introduction} %Done!


Over the years, the vehicle routing problem has gained many extensions and formal definitions due to its numerous real-world applications. The basic version of the problem involves finding a distance-optimal set of routes for a fleet of identical vehicles serving a set of customers, with the objective of minimizing the total travel distance or time. Despite the conceptual simplicity of this base model, its extensions and adaptations have proven essential to represent real-life logistics and service-related challenges with more precision. An example would be the incorporation of various operational constraints, such as time windows, heterogeneous fleets, or multi-depot structures, to better reflect practical scenarios. In recent years, increasing attention has been given to hybrid and multi-criteria variants of the vehicle routing problem, which integrate economic, environmental, and human-resource-related aspects into the optimization process \cite{vidal2013heuristics}. These developments have significantly broadened the applicability of VRP-based models, extending their use beyond traditional logistics to areas such as service management, healthcare logistics, and infrastructure maintenance. Consequently, the flexibility of the VRP formulations enables them to serve as a solid foundation for solving complex resource allocation and scheduling problems in various industrial sectors \cite{pillac2013review}.

One proposed application of the routing problem is to plan routes for crews servicing building automation equipment. Nowadays, virtually every new building must have amenities such as fire protection systems, HVAC (Heating, Ventilation, and Air Conditioning) systems, and personnel movement devices. These installations require regular inspections and maintenance to ensure operational safety and compliance with legal standards. Such inspections are usually performed by mobile technician teams, each with specific qualifications, level of experience, and associated service rates \cite{Francis}. In addition, each visit requires a certain amount of time depending on the type of repair and the service materials transported by the crews. Planning their work must take into account aspects such as limited operating hours, limited vehicle capacity, and varying rates based on employee qualifications. In turn, planning the order of visiting individual locations must take into account the availability of the location (time window), the location of the building and the required service time. These conditions lead to the formulation of a Heterogeneous Crew Vehicle Routing Problem with Time Windows (HCVRP-TW).

In this paper, a mathematical model for the HCVRP-TW problem is formulated using MILP. The work is an extension of the approach studied in \cite{Pempera} and \cite{Idzikowski}. Several solving methods are implemented and compared, including a commercial MILP solver, an adapted Tabu Search algorithm, a simple 1-nearest neighbor (1NN) heuristic, and the Artificial Bee Colony algorithm. The main objectives of this study are fivefold:

\begin{enumerate}
    \item [\textbf{PG1:}] Check the status of research on HCVRP-TW problems, especially in application to building maintenance.
    \item [\textbf{PG2:}] Propose and describe the MILP model for the given HCVRP-TW problem.
    \item [\textbf{PG3:}] Verify if the problem posed can be efficiently solved using commercial MILP solvers.
    \item [\textbf{PG4:}] Measure and compare the performance of local search algorithms (Tabu Search) and swarm algorithms (Artificial Bee Colony).
    \item [\textbf{PG5:}] Find the impact of time windows length and the number of more qualified teams to the problem solution.
\end{enumerate}
The remainder of this paper is organized as follows. Section \ref{sec:literature} provides a detailed literature review of existing approaches to HCVRP-TW and related problems. Section \ref{sec:problem} introduces the problem formulation, model assumptions, and constraints. Section \ref{sec:experiment} presents the computational experiments, followed by the analysis of the results. Finally, Section \ref{sec:conclusion} summarizes the findings and outlines the directions for future work.




% In practical scenarios, the VRP often incorporates various additional constraints, such as vehicle capacity limitations, service time requirements, heterogeneous vehicle fleets, and specific time windows during which customers can be visited. These extensions have given rise to several well-known variants, including the Capacitated VRP (CVRP), VRP with Time Windows (VRPTW), and Heterogeneous Fleet VRP (HFVRP). Each variant reflects a different set of operational challenges encountered in logistics, transportation, and service industries. Combining several of these constraints in one model increases both realism and computational complexity, making such problems NP-hard and difficult to solve exactly for larger instances.